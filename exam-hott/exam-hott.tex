\documentclass[12pt]{article}

\usepackage{a4wide}
\usepackage[T1]{fontenc}
\usepackage[utf8]{inputenc}
\usepackage{times}
\usepackage{amsmath}
\usepackage{amssymb}
\usepackage{amsthm}

{
\theoremstyle{definition}
\newtheorem{problem}{Problem}
}

\newcommand{\dsum}[1]{\Sigma (#1) \,.\,}
\newcommand{\dprod}[1]{\Pi (#1) \,.\,}
\newcommand{\univ}{\mathcal{U}}
\newcommand{\susp}[1]{\mathsf{Susp}(#1)}
\newcommand{\two}{\mathsf{2}}
\newcommand{\eqv}{\simeq}

\begin{document}

\title{Homotopy (type) theory exam -- Part II}
\author{}
\date{}
\maketitle

\section*{Instructions}

For full credit in this part of the course, solve \emph{at least five problems}.

As you are training to become a researcher, you are free to refer to constructions and
proofs in existing literature, namely peer-reviewed papers and monographs. References to
blog posts and other non-standard sources are allowed, but in those cases you need to
verify the veracity of the claims yourself. It is probably a good idea to verify your
sources even when they are of a reputable origin. In the end, you are responsible for your
solutions.

\section*{Problems}

\begin{problem}
  Prove that the coproducts have the expected universal property:
  %
  \begin{equation*}
    (A + B \to C) \eqv (A \to C) \times (B \to C).
  \end{equation*}
\end{problem}

\begin{problem}
  Let $A$ be a type and $a : A$ a point. Prove that $\dsum{x : A} a =_A x$ is contractible.
\end{problem}

\begin{problem}
  Prove that $\mathbb{N}$ is a set.
\end{problem}

\begin{problem}
  Show that $(\two \eqv \two) \eqv \two$.
\end{problem}

\begin{problem}
  Show that $S^1 \eqv \susp{\two}$, where $S^1$ is the circle and $\susp{\two}$ the
  suspension of~$\two$.
\end{problem}

\begin{problem}
  How would you define the \emph{double cover} of the circle as a dependent type? That is,
  construct a dependent type $D : S^1 \to \univ$ such that $D(\mathsf{base}) \eqv \two$
  and $(\dsum{x : S^1} D(x)) \eqv S^1$.
\end{problem}

\begin{problem}
  How would you define the \emph{Möbius band} as a type?
\end{problem}

\end{document}
