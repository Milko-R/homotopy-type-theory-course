\documentclass{article}

\usepackage[T1]{fontenc}
\usepackage[utf8]{inputenc}
\usepackage{times}
\usepackage{amsmath}
\usepackage{amssymb}
\usepackage[hidelinks]{hyperref}
\usepackage{mathpartir}
\usepackage{booktabs}

\newcommand{\defeq}{\mathrel{{:}{=}}}

\newcommand{\Empty}{\mathsf{0}} % the empty type
\newcommand{\absurd}[2]{\mathsf{absurd}_{#1}(#2)} % the eliminator for empty

\newcommand{\Unit}{\mathsf{1}} % the unit type
\newcommand{\pt}{\star} % the element of the unit type

\newcommand{\Bool}{\mathsf{2}} % the boolean type
\newcommand{\fal}{\mathsf{0}_\Bool}
\newcommand{\tru}{\mathsf{1}_\Bool}
\newcommand{\cond}[3]{\mathsf{if}\;#1\;\mathsf{then}\;#2\;\mathsf{else}\;#3} % conditional

\newcommand{\Nat}{\mathsf{N}} % the natural numbers
\newcommand{\zero}{\mathsf{0}} % zero
\renewcommand{\succ}[1]{\mathsf{S}\,#1} % successor

\newcommand{\ind}[1]{\mathsf{ind}_{#1}} % induction principle

\newcommand{\Univ}{\mathsf{U}} % the universe

\newcommand{\dprod}[1]{\Pi(#1) \,.\,} % dependent product
\newcommand{\lam}[1]{\lambda (#1) \,.\,} % lambda abstraction
\newcommand{\lamx}[1]{\lambda #1 \,.\,} % lambda abstraction

\newcommand{\dsum}[1]{\Sigma(#1) \,.\,} % dependent sum

\newcommand{\istype}[1]{#1\; \mathsf{type}} % is a type

\begin{document}
\title{Lecture 1: Type theory}
\author{Andrej Bauer}
\maketitle

\begin{abstract}
  Lecture notes for the second part of the doctoral course ``Homotopy (Type)
  Theory'', taught by Jaka Smrekar and Andrej Bauer in the Spring 2019 at the
  Department of Mathematics, Faculty of Mathematics and Physics, University of
  Ljubljana.
\end{abstract}


\section{What is type theory?}
\label{sec:what-type-theory}

In this lecture we give an introduction to type theory aimed at mathematicians
who are possibly unacquainted with logic and theoretical computer science. In
fact, the whole series of lectures will consciously avoid discussing the
connections between type theory and computer science.

\subsection{Type theory as a foundation}

Type theory is a \emph{foundational theory}. This means that it stands on its
own, without any support from logic and set theory, and it aims to be general
enough to encompass a great deal of mathematical activity. While this sounds a
bit scary, it also means that type theory can in principle be understood and
used without prior knowledge of algebra, category theory, analysis, etc. The
situation is not unlike many others in mathematics, where we limit ourselves to
a certain style of reasoning. For example, when doing planar geometry we only
use the basic geometric objects (points, segments, lines, circles) and we adhere
to the axioms of geometry.

Type theory is not a replacement, or an alternative, to logic and set theory.
Rather, it is a \emph{generalization} of existing foundations that
\emph{expands} the view of the mathematical universe. In later lectures we shall
identify logic and sets as just two homotopy levels of the type-theoretic
mathematical universe.

There are several variants of type theory. We shall focus on \emph{homotopy type
  theory (HoTT)}, a recently developed version which has close connections to
abstract and sophisticated topics, such as $\infty$-groupoids and
$(\infty,1)$-toposes. However, we will steer clear of a precise technical
explanation on how type theory relates to these, as that would require a course
on its own. Instead, we will explain types intuitively, relying on your prior
mathematical knowledge and intuitions about geometry and space.

Of course, type theory itself will be formulated quite precisely and as
rigorously as any other mathematics (and more so, because it has all been
formalized and verified by computer programs known as proof assistants).


\subsection{A theory of constructions}

In type theory, we take as the central notion \emph{construction} of
mathematical objects. Constructions permeate mathematics, and are at least as
important as proofs. In fact proofs themselves are a certain kind of
constructions: we construct proofs from axioms using the logical rules of
reasoning.

By taking constructions as primitive, we are in good company. Proposition~1 of
the First book of Euclid's Elements~\cite{fitzpatrick09:_euclid_elemen} reads:
%
\begin{quote}
  \textit{To construct an equilateral triangle on a given finite straight-line.}
\end{quote}
%
Mathematics started with a construction problem, not a statement of truth!

Euclid's Proposition~1 is not just any kind of construction but rather a
\emph{dependent} one: the constructed equilateral triangle depends on the line
segment. Dependence is a basic phenomenon in mathematics, thus our theory will
be a theory of \emph{dependent constructions}.

\section{The basics of type theory}

Let us proceed with an overview of the basic concepts of type theory, namely
\emph{types} and \emph{points} of types.

A type is a description of a certain kind of construction. For example, we
will introduce the type of natural numbers~$\Nat$ by explaining how to
construct natural numbers, and how to use them. When an object is constructed
according to these rules, we say that its type is~$\Nat$. In fact, every
construction must always be made according to the rules of one of the types, and
then it is of course of that type. It cannot happen that a construction is of
two different types, for each type will prescribe a different way of building
objects.

It is important to distinguish a type from its \emph{extension}. A type
describes a way of constructing objects. The extension of a type is the
collection (we avoid saying ``set'' lest we suggest unwarranted intuitions) of
all things that may be constructed according to the given rules. It may happen
that two different types have the same extension. This already happens quite
naturally in informal mathematics. Consider:
%
\begin{enumerate}
\item[($P$)] the even primes larger than three,
\item[($L$)] the straight lines in the plane that are perpendicular to themselves.
\end{enumerate}
%
There are no even primes larger than three, and no straight lines perpendicular
to themselves, so the extensions of~$P$ and~$L$ are both vacuous. However, it
would be absurd to say that the meanings of ``even prime larger than three'' and
``straigh line perpendicular to itself'' coincide!

If $A$ is a type and $t$ is a construction made according to the rules of~$A$,
then we write
%
\begin{equation*}
  t : A.
\end{equation*}
%
We call $t$ an \emph{element} or a \emph{point} of~$A$. The word ``element''
makes one think that~$A$ is a collection of things (which it is not, but its
extension is), while ``point'' emphasizes the view that types are like spaces.

People close to logic, as well as theoretical computer scientists, sometimes say
that~$t$ is a \emph{term} of type~$A$, as if constructions were just formal
symbolic expressions. We eschew this view, because it is analogous to thinking
that a function is a symbolic expression with the letter~$x$ appearing in it.

Just making constructions without ever doing anything with them would lead to
pretty boring mathematics. We should have the ability to \emph{deconstruct} and
\emph{use} constructed objects. Thus in type theory there are generally two
kinds of operations, the constructors and the deconstructors.

In addition, there may be several ways of constructing the same object, for
example by undoing some construction steps (using deconstructors), and then
redoing them. Consequently, we should be able to compare the constructed
objects. For this purpose we introduce \emph{equations}. An equation
%
\begin{equation*}
  s \equiv_A t,
\end{equation*}
%
states that points $s$ and $t$ of type~$A$ are equal.
%
Similarly, we may state that two types $A$ and $B$ are the same, written as
%
\begin{equation*}
  A \equiv B.
\end{equation*}
%
It only makes sense to compare points of the same type. That is, if $s : A$ and
$t : B$ then $s \equiv_A t$ is not even false --- it is senseless, unless we can
establish $A \equiv B$.

Equality satisfies the usual rules of manipulation: reflexivity, symmetry,
transitivity, equals may be subsituted for equals. But since there is no logic,
we cannot deny an equality: there is no such thing as $\lnot (A \equiv B)$ or
$A \not\equiv B$. The situation is similar to universal algebra, where
structures are described in terms of operations and equations only.

There is one last ingridient that we need to address, namely \emph{dependency}.
Everywhere in mathematics we build objects and spaces which depend on
parameters. An example from ordinary mathematics: the space of real-valued
continuous maps $\mathcal{C}([a,b], \mathbb{R})$ is not really a single space. It
is a \emph{family} of spaces, one for each pair of real numbers $(a,b)$ such
that $a < b$. We say that $\mathcal{C}([a,b], \mathbb{R})$ \emph{depends} on the
parameters~$a$ and~$b$. Let us also note that $\mathbb{R}$ is \emph{not} a
parameter, but instead a fixed space that has been previously constructed.

We shall write
%
\begin{equation*}
  \istype{A}
\end{equation*}
%
to indicate a type~$A$. A type which depends on a parameter $x : A$ is indicated by
%
\begin{equation*}
  x : A \vdash \istype{B(x)}
\end{equation*}
%
More generally,
%
\begin{equation*}
  x_1 : A_1, x_2 : A_2(x_1), \ldots, x_n : A_n(x_1, \ldots, x_{n-1}) \vdash
  \istype{B(x_1, \ldots, x_n)}
\end{equation*}
%
signfies a type depending on several parameters $x_1, \ldots, x_n$.

Not only do types depend on parameters, but so do their points. We write
%
\begin{equation*}
  x : A \vdash s(x) : B(x).
\end{equation*}
%
when $s(x)$ is a point of type $B(x)$, with both the point and the type
depending on a parameter~$x : A$. Perhaps an informal example will help: both
the interval $[a,b]$ and its midpoint $(a + b)/2$ depends on real-valued
parameters $a$ and $b$.

In ordinary mathematics there is no standard notation for indicating precisely
on which parameters a given construction depends. Mathematical prose relies on
the reader's ability to determine the dependencies from the context. We shall do
so as well, except when we need to be precise about the rules of type theory.

To summarize, the basic building blocks of type theory are as follows:
%
\begin{center}
\begin{tabular}{lc}
  \toprule
  \textsc{Concept} & \textsc{Notation} \\
  \midrule
  type             & $\istype{A}$ \\
  point            & $t : A$ \\
  dependent type   & $x : A \vdash \istype{B(x)}$ \\
  dependent point  & $x : A \vdash s(x) : B(x)$ \\
  equal types      & $A \equiv B$ \\
  equal points     & $s \equiv_A t$ \\
  \bottomrule
\end{tabular}
\end{center}
%
There is nothing else. Stating that $t$ does \emph{not} have type~$A$ is not
part of type theory, nor is it part of type theory to consider a point~$t$
without a type. You will have to take it on faith that these restrictions exist
for good reasons, and that eventually they will allow us to reap some benefits.
(Compare the situation to universal algebra: restricting the axioms to be just
equalities is, well, restrictive, but the resulting theory is still very rich.)

\subsection{Basic type-theoretic constructions}

It is high time we actually introduce some specific types. Let us begin with the
very basic ones. With these we shall not be able to construct anything that
would excite homotopy theorists, say a circle or a suspension. We need to learn
to walk before we can run.

There is an established method for introducing a new type:
%
\begin{itemize}
\item \emph{formation:} we postulate rules which explain how the new type is made
\item \emph{construction:} we postulate rules for constructing the points of the new type
\item \emph{deconstruction:} we postulate rules for deconstructing or using the points of the new type
\item \emph{equations:} we postulate equations which explain what happens when a
  construction is followed by a deconstruction, or vice versa.
\end{itemize}
%

\subsubsection{The unit type $\Unit$}

Our first example is the \emph{unit} type. Intuitively speaking, it corresponds to a one-point space. Here are the rules:
%
\begin{itemize}
\item formation: $\Unit$ is a type,
\item construction: $\pt$ is a point of $\Unit$,
\item deconstruction: there are no deconstructors,
\item equations: any two points $s$ and $t$ of type $\Unit$ are equal.
\end{itemize}
%
There are no deconstructors because nothing useful can be extracted from~$\pt$.
Or to put it another way, $\pt$ has no internal structure, hence there is
nothing to deconstruct.

Type theorists would write the rules for the unit type as follows:
%
\begin{mathpar}
  \infer{ }{\istype{\Unit}}
  \and
  \infer{ }{\pt : \Unit}
  \and
  \infer{s : \Unit \\ t : \Unit}{s \equiv_\Unit t}
\end{mathpar}
%
Such ``fractions'' should be read as inference rules: above the line are the
\emph{premises}, and below the line is the \emph{conclusion}. We only
occasionally use inference rules, but be warned that hard-core type theorists
thrive on writing pages and pages of inference rules, and they can read them
really fast, too.

\subsubsection{The empty type $\Empty$}

Next we have the empty type $\Empty$, which intuitively corresponds to the empty
space. Thus, not suprisingly, it will have no constructors. But how do we say
that there is nothing in it? Here are the rules:
%
\begin{itemize}
\item formation: $\Empty$ is a type,
\item construction: there are no constructors
\item deconstruction: for every $e : \Empty$ and a type~$A$ there is a point $\absurd{A}{e} : A$.
\item equations: if $e : \Empty$, then any two points of any type are equal.
\end{itemize}
%
Written as inference rules, these are:
%
\begin{mathpar}
  \infer{ }{\istype{\Empty}}
  \and
  \infer{e : \Empty \\ \istype{A}}{\absurd{A}{e} : A}
  \and
  \infer{e : \Empty \\ s : A \\ t : A}{s \equiv_A t}
\end{mathpar}
%
If you know a little bit of category theory you should draw an analogy with the
initial object: the deconstructor $\absurd{A}{t}$ gives us a way of constructing
a point of any type~$A$ from a point $e : \Empty$. This is the type-theoretic
way of saying that ``$\Empty$ is as empty as any other type''. The equation rule
expresses the absurdity of having a point in~$\Empty$, at the level of
equations.

\subsubsection{The booleans $\Bool$}

The type of $\Bool$ is a space with two distinct points. We thus have:
%
\begin{itemize}
\item formation: $\Bool$ is a type,
\item constructors: $\tru$ and $\fal$ are points of $\Bool$.
\end{itemize}
%
We carefully subscripted the points $\tru$ and $\fal$ with ${}_\Bool$ in order
to distinguish them from the empty and the unit type, and from the natural
numbers zero and one. We must do so because a point can only ever have one type.
When such formalism becomes cumbersome we shall lapse into using the same symbol
for several different points, but until then let us be very explicit.

The deconstructor and the equations for $\Bool$ are as follows: given a
dependent type $x : \Bool \vdash \istype{A(x)}$, points $s : A(\fal)$ and
$t : A(\tru)$, and a point $b : \Bool$, there is a point
%
\begin{equation}
  \label{eq:ind-bool}%
  \ind{\Bool}^A(b, s, t) : A(b)
\end{equation}
%
such that
%
\begin{align*}
  \ind{\Bool}^A(\fal, s, t) &\equiv_{A(\fal)} s, \\
  \ind{\Bool}^A(\tru, s, t) &\equiv_{A(\tru)} t.
\end{align*}
%
Two alternative, but less explicit notations for~\eqref{eq:ind-bool} are
%
\begin{equation*}
  \cond{b}{s}{t}
  \qquad
  \qquad
  \text{and}
  \qquad
  \qquad
  \begin{cases}
    s & \text{if $b \equiv_\Bool \fal$,} \\
    t & \text{if $b \equiv_\Bool \tru$.}
  \end{cases}
\end{equation*}
%



\subsubsection{Dependent products $\dprod{x : A} B(x)$}

\subsubsection{The natural numbers $\Nat$}

\subsubsection{Dependent sums $\dsum{x : A} B(x)$}

\subsubsection{The universe $\Univ$}


\subsection{Derived constructions}

\subsubsection{Exponentials $A \to B$}

\subsubsection{Simple products $A \times B$}

\subsubsection{Sums $A + B$}

\subsection{What about $\cap$ and $\cup$?}


\section{The homotopy-theoretic reading of types}

So far we have been dealing mostly with pure formalism. Is there a way to
explain what the types really are? Yes, there are several ways! Type theory is
so general that it can be understood in many ways: types are datatypes in a
programming language, types are sets, types are symmetric transitive relations
on algebraic lattices, types are spaces. It is this last explanation which we
will rely on and that we now briefly outline.

Keep in mind that what follows is an informal explanation of how to imagine
types. We are not defining anything, nor are we introducing any new mathematics.
We are just going to draw helpful pictures.

Let us augment our table of basic concepts with the homotopy-theoretic explanation:
%
\begin{center}
\begin{tabular}{lcc}
  \toprule
  \textsc{Concept} & \textsc{Notation} & \textsc{Homotopy theory} \\
  \midrule
  type             & $\istype{A}$ & space \\
  point            & $t : A$ & point \\
  dependent type   & $x : A \vdash B(x)$ & fibration $B$ over $A$\\
  dependent point  & $x : A \vdash s(x) : B(x)$ & section of $B$ \\
  equal types      & $A \equiv B$ & equal spaces \\
  equal points     & $s \equiv_A t$ & equal points  \\
  \bottomrule
\end{tabular}
\end{center}
%
The most revealing is the interpretation of dependent type: a type $B(x)$
depending on $x : A$ is a fibration with base $A$ whose fiber at~$x$ is the
space~$B(x)$.

We may proceed with the basic constructions:
%
\begin{center}
\begin{tabular}{lcc}
  \toprule
  \textsc{Concept} & \textsc{Notation} & \textsc{Homotopy theory} \\
  \midrule
  unit type        & $\Unit$    & one-point space \\
  empty type       & $\Empty$   & empty space \\
  boolean type     & $\Bool$   & two-point discrete space \\
  natural numbers  & $\Nat$     & discrete space $\mathbb{N}$ \\
  dependent sum    & $\dsum{x : A} B(x)$ & total space \\
  dependent product & $\dprod{x : A} B(x)$ & space of sections \\
  \bottomrule
\end{tabular}
\end{center}
%
All of these should be accompanied with nice pictures. For instance, the
induction principle for~$\Nat$ can be seen as giving a way of producing a
section of a fibration over~$\Nat$.

You will notice that we have not given a homotopy-theoretic interpretation of
the universe~$\Univ$. What would that be? A space of all spaces, presumably, but
have we ever seen such a thing in homotopy theory? Things are going to get
interesting.

\bibliographystyle{plain}
\bibliography{notes}

\end{document}


